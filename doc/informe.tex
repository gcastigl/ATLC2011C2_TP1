\documentclass[a4paper,10pt]{article}

\usepackage[utf8]{inputenc}
\usepackage{t1enc}
\usepackage[spanish]{babel}
\usepackage[pdftex,usenames,dvipsnames]{color}
\usepackage[pdftex]{graphicx}
\usepackage{amsmath}
\usepackage{amsfonts}
\usepackage{amssymb}
\usepackage[table]{xcolor}
\usepackage[small,bf]{caption}
\usepackage{float}
\usepackage{subfig}
\usepackage{listings}
\usepackage{bm}
\usepackage{times}

\setcounter{secnumdepth}{5}

\begin{document}

\begin{titlepage}
	\thispagestyle{empty}
	\begin{center}
		\includegraphics[scale=0.7]{./images/itba.jpg}
		\vfill
		\Huge{Autómatas, Teoria de Lenguajes y Compiladores}\\
		\vspace{1cm}
		\huge{Trabajo Práctico Especial 1} \\
		\vspace{0.3cm}
		\huge{Title}
	\end{center}
	\vspace{2cm}
	\large{
		\begin{tabular}{lcr}
			Castiglione, Gonzalo & & 49138 \\
			Susnisky, Darío & & 50592 \\
			Ordano, Esteban & & 50753 \\
			Sturla, Martín & & 50684 \\
			\\ 
		\end{tabular}
	}
	\vfill
	\flushright{\today}
\end{titlepage}

\newpage

%%%%%%%%%%%%%%%%%%%%%%%%%%%%%%%%%%
%%%%%%%%% begin CONTENT %%%%%%%%%%
%%%%%%%%%%%%%%%%%%%%%%%%%%%%%%%%%%

	\thispagestyle{empty}
\tableofcontents

\newpage

\setcounter{page}{1}

\newpage

\section{Resumen}
El trabajo práctico consistia en poder leer tanto un autómata como una gramática y transformarlo en su contraparte
 equivalente. Para esto, era necesario leer e interpretar los archivos de entrada con archivos de \textit{lex}.

\newpage

\section{Consideraciones realizadas}
    No hubo consideraciones externas hechas ya que en caso de errores en los archivos de entradas era nuestro trabajo
     detectarlos. Por otra parte, dentro de la lógica del programa si se valida que la gramática sea coherente 
      (por ejemplo que no hayan caracteres repetidos entre los símbolos terminales y los no terminales) y regular
      (en caso que el archivo de entrada sea una gramática). En caso de que el archivo de entrada sea un autómata
      bastaba con chequear su validez como autómata.

\newpage

\section{Descripción del desarrollo}
    Al comienzo, pudimos detectar distintos módulos del trábajo. Estos implicaban tanto leer un autómata o una gramática
     y dejar sus datos de forma accesible en las estructuras correctas. Estos \textit{parsers} eran los encargados
      de validar los datos de entrada.
      
      Luego contabamos con los módulos que manipulaban las estructuras, pasando de autómata a gramática y viseversa.
      También era necesario un algoritmo que transforme cualquier tipo de gramáticas regulares en gramáticas 
       regulares derechas.

      Por último, existian los modulos de \textit{output} que debían contemplar tanto el caso en que se quiera imprimir
       un autómata como una grámatica.
      El siguiente gráfico muestra la la estructura que pensamos inicialmente como nuestro programa.

      PONER GRAFICO

      Luego de un segundo análisis comprendimos que no era necesario contar con una estructura que represente
       un autómata, pues era sencillo hacer la conversión entre ambas partes sin contar con esta estructura.
      Así, se formo este segundo diagrama:

      PONER GRAFICO

      Al llegar al final del trabajo, comprendimos que por la manera en que era pedida la muestra de datos,
       era cómodo contar con la estructura que representaba al autómata. Así, nuestra versión final es bastante similar
       a la primera.

      A continuación se realiza un análisis un poco más profundo en cada uno de los modulos.

      \newpage

      \subsection{Parsers}
            Para realizar la interpretación de los archivos de entrada, primero fue necesario familiarizarse con
            \textit{lex}. Esto era muy importante ya que no podían quedar reglas sin especificar. 
            Luego, fue esencial modelar el \textit{parser} como una maquina de estados. De este modo y teniendo
            todos los estados posibles identificados fue sencillo plasmar código de C que realice lo que cada caso
            requería. 
            Estos procesos fueron llevados a cabo tanto para leer un autómata o una gramática.
      \subsection{Conversión entre estructuras}
            Dados los algoritmos vistos en clase y ciertas pruebas que hicimos fue sencillo ubicar los algoritmos
            para convertir autómatas en gramáticas y viseversa.

            FALTA EXPLICAR BIEN CADA UNO
      \subsection{Conversion de gramática a gramática alineada a derecha}
            QUE LO HAGA ALGUIEN QUE SEPA BIEN QUE HACE EL ALGORITMO
      \subsection{Output}
            El \textit{output} no presentó una complicación mayor ya que al contar con los formatos de salida 
            requeridos y al tener los datos ordenados en estructuras correctas, el proceso fue casi intuitivo.
            Por otra parte hubo que tener cuidado con el uso de \textit{graphbiz}, pero acompañados de ciertos
            conocimientos previos esto tampoco presento una grán complicación.

\newpage

\section{Dificultades encontradas}
    Más alla de los cambios que hubo que ir haciendo en el diseño de la aplicación y algún que otro \textit{bug} 
    , podemos decir que no nos topamos con dificultades muy grandes. Igualmente, es interesante notar hubo que 
    familiarizarse con la sintaxis de \textit{lex} y que el algoritmo para transformar una gramática en una 
    gramática alineada a dereche no fue sencillo de implementar.
\newpage

\section{Futuras Extensiones}
    HAHAHAHA, SERIOUSLY? HAAHHAAHAHAHAH
\end{document}

